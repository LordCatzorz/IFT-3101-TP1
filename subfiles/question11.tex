% !TeX root = ../main.tex
\documentclass[class=article]{standalone}

\begin{document}
\section*{Question 11}

Calcul de l'ensemble \lstinline{FIRST}(S)

\begin{deriv}
    \text{\lstinline{FIRST}}(S)
    \<=
    \commentaire{Toutes les productions de $S$}
    \text{\lstinline{FIRST}}(\epsilon) \, \cup \,
        \text{\lstinline{FIRST}}(aSaSb) \, \cup \,
        \text{\lstinline{FIRST}}(aSbSa) \, \cup \,
        \text{\lstinline{FIRST}}(bSaSa) \, \cup \,
        \text{\lstinline{FIRST}}(SS)
    \<=
    \commentaire{\lstinline{FIRST}$(SS) =$ \lstinline{FIRST}$(S)$}
    \text{\lstinline{FIRST}}(\epsilon) \, \cup \,
        \text{\lstinline{FIRST}}(aSaSb) \, \cup \,
        \text{\lstinline{FIRST}}(aSbSa) \, \cup \,
        \text{\lstinline{FIRST}}(bSaSa)
    \<=
    \commentaire{\lstinline{FIRST}$(\epsilon) = \acc{\epsilon}$}
    \acc{\epsilon} \, \cup \,
        \text{\lstinline{FIRST}}(aSaSb) \, \cup \,
        \text{\lstinline{FIRST}}(aSbSa) \, \cup \,
        \text{\lstinline{FIRST}}(bSaSa)
    \<=
    \commentaire{\lstinline{FIRST}$(a\alpha) = \acc{a}$}
    \acc{\epsilon} \, \cup \,
        \acc{a} \, \cup \,
        \acc{a} \, \cup \,
        \text{\lstinline{FIRST}}(bSaSa)
    \<=
    \commentaire{\lstinline{FIRST}$(b\alpha) = \acc{b}$}
    \acc{\epsilon} \, \cup \,
        \acc{a} \, \cup \,
        \acc{a} \, \cup \,
        \acc{b}
    \<=
    \commentaire{Union}
    \acc{a \, , b \, , \epsilon}
\end{deriv}

Calcul de l'ensemble \lstinline{FOLLOW}(S)

La première règle. $S$ est le symbole de départ.
\begin{center}
\lstinline{FOLLOW}$(S) \supseteq \acc{\$}$
\end{center}

Calcul des $2^e$ règles

\begin{center}
    \begin{tabular}{|rl|rlllll|}
        \hline
        \multicolumn{2}{|c}{\bf Symbole} &
        \multicolumn{6}{|c|}{\bf Contraintes identifiées}\\
        \hline
        \hline
        $S$ & $\rightarrow a\underline{S}aSb$ & $\text{\lstinline{FOLLOW}}(S)$ & $\supseteq \text{\lstinline{FIRST}}(a)$ & $- \acc{\epsilon}$ & $= \acc{a}$ & $- \acc{\epsilon}$ & $= \acc{a}$\\
        $S$ & $\rightarrow aSa\underline{S}b$ & $\text{\lstinline{FOLLOW}}(S)$ & $\supseteq \text{\lstinline{FIRST}}(b)$ & $- \acc{\epsilon}$ & $= \acc{b}$ & $- \acc{\epsilon}$ & $= \acc{b}$\\
        \hline
        $S$ & $\rightarrow a\underline{S}bSa$ & $\text{\lstinline{FOLLOW}}(S)$ & $\supseteq \text{\lstinline{FIRST}}(b)$ & $- \acc{\epsilon}$ & $= \acc{b}$ & $- \acc{\epsilon}$ & $= \acc{b}$\\
        $S$ & $\rightarrow aSb\underline{S}a$ & $\text{\lstinline{FOLLOW}}(S)$ & $\supseteq \text{\lstinline{FIRST}}(a)$ & $- \acc{\epsilon}$ & $= \acc{a}$ & $- \acc{\epsilon}$ & $= \acc{a}$\\
        \hline
        $S$ & $\rightarrow b\underline{S}aSa$ & $\text{\lstinline{FOLLOW}}(S)$ & $\supseteq \text{\lstinline{FIRST}}(a)$ & $- \acc{\epsilon}$ & $= \acc{a}$ & $- \acc{\epsilon}$ & $= \acc{a}$\\
        $S$ & $\rightarrow bSa\underline{S}a$ & $\text{\lstinline{FOLLOW}}(S)$ & $\supseteq \text{\lstinline{FIRST}}(a)$ & $- \acc{\epsilon}$ & $= \acc{a}$ & $- \acc{\epsilon}$ & $= \acc{a}$\\
        \hline
        $S$ & $\rightarrow \underline{S}S$ & $\text{\lstinline{FOLLOW}}(S)$ & $\supseteq \text{\lstinline{FIRST}}(S)$         & $- \acc{\epsilon}$ & $= \acc{a \, , b \, , \epsilon}$ & $- \acc{\epsilon}$ & $= \acc{a \, , b}$\\
        $S$ & $\rightarrow S\underline{S}$ & $\text{\lstinline{FOLLOW}}(S)$ & $\supseteq \text{\lstinline{FIRST}}(\epsilon)$ & $- \acc{\epsilon}$  & $= \acc{\epsilon}$                & $- \acc{\epsilon}$ & $= \acc{}$\\
        \hline
    \end{tabular}
\end{center}
\pagebreak
Calcul des $3^e$ règles
\begin{center}
    \begin{tabular}{|rl|l|rl|}
        \hline
        \multicolumn{2}{|c}{\bf Symbole} &
        \multicolumn{1}{|c}{\bf Justification} &
        \multicolumn{2}{|c|}{\bf Contraintes identifiées}\\
        \hline
        \hline
        $S$ & $\rightarrow a\underline{S}aSb$ & - & - & \\
        $S$ & $\rightarrow aSa\underline{S}b$ & - & - &\\
        \hline
        $S$ & $\rightarrow a\underline{S}bSa$ & - & - &\\
        $S$ & $\rightarrow aSb\underline{S}a$ & - & - &\\
        \hline
        $S$ & $\rightarrow b\underline{S}aSa$ & - & - &\\
        $S$ & $\rightarrow bSa\underline{S}a$ & - & - &\\
        \hline
        $S$ & $\rightarrow \underline{S}S$ & Suivi d'un symbole annulable & $\text{\lstinline{FOLLOW}}(S)$ & $\supseteq \text{\lstinline{FOLLOW}}(S)$\\
        $S$ & $\rightarrow S\underline{S}$ & Situé à la fin & $\text{\lstinline{FOLLOW}}(S)$ & $\supseteq \text{\lstinline{FOLLOW}}(S)$\\
        \hline
    \end{tabular}
\end{center}

Plus petite solution:

$\text{\lstinline{FOLLOW}}(S) = \acc{a \, , b \, ,\$}$
\end{document}