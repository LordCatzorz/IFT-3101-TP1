% !TeX root = ../main.tex
\documentclass[class=article]{standalone}

\begin{document}
\section*{Question 5}

Toutes les chaînes de $L_5$ sont généré par $G_5$. Preuve par induction sur le nombre de dérivations.

{\bf Case de base}

Une dérivation de $S$ en une étape a bien deux fois plus de $a$ que de $b$. 

La preuve, c'est que la seule dérivation possible en une 
étape est $S \rightarrow \epsilon$. Qui contient bien deux fois 
plus de $a$ que de $b$: 

$|\epsilon|_a - 2|\epsilon|_b = 0 - 2\cdot0 = 0$,
donc il y a 2 fois plus de $a$ que de $b$.

{\bf Hypothèse d'induction}

Supposons que toutes les chaînes répondant à la spécification de $L_5$
de longueur inférieure à $3n$, pour $n \geq 1$, sont générer par $G_5$,
alors toute chaîne $w$ de longueur égale à $3n$ répondant à la spécification de $L_5$
est aussi générée par $G_5$, 
(Notons aussi que toutes les chaînes de $L_5$ 
sont de longueur multiple de 3). 

La preuve, 

Si $w$ commence par un $b$, alors deux $a$ doivent 
le suivre selon la seule dérivation permettant un $b$ au début: $S \rightarrow bSaSa$.
Alors $w$ peut être écrite sous la forme 
$w = bxayaz$, où $x$, $y$ et $z$ 
sont tous de longueur inférieure à $3n$ et dont leur concaténation
est de longueur $3n-3$ 
et donc répondent aux critères de $L_5$.


Si $w$ fini par un $b$, alors deux $a$ doivent 
le précédé selon la seule dérivation permettant un $b$ à la fin: $S \rightarrow aSaSb$.
Alors $w$ peut être écrite sous la forme 
$w = xaxayb$, où $x$, $y$ et $z$
sont tous de longueur inférieure à $3n$ et dont leur concaténation
est de longueur $3n-3$ et donc répondent aux critères de $L_5$.

Si $w$ commence et fini par un $a$, alors un $b$ doit se trouver entre
les deux selon la dérivation $S \rightarrow aSbSa$.
Alors $w$ peut être écrit sous la forme
$axbyz$ où $a$ et $y$
sont tous de longueur inférieure à $3n$ et dont leur concaténation
est de longueur $3n-3$ et donc répondent aux critères de $L_5$.

Tous les cas possibles de $w$ ont été énumérés, donc $L_5 \subseteq L(G_5)$



\end{document}